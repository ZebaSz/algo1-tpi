\documentclass[a4paper]{article}

\usepackage[spanish]{babel} % Le indicamos a LaTeX que vamos a escribir en espa�ol.
\usepackage[latin1]{inputenc} % Permite utilizar tildes y e�es normalmente
\input{Algo1Macros}% Macros especificas para especificar problemas en AyEDI
\usepackage{caratula} % Se puede descargar en ~> https://github.com/bcardiff/dc-tex

\usepackage{listings} % Para poner codigo
\lstset{language=C++, breaklines=true, showstringspaces=false, mathescape=true, fontadjust, tabsize=4}

% Aca solo vamos a poner el esqueleto del documento, pero no vamos a especificar nada.

\begin{document} % Todo lo que escribamos a partir de aca va a aparecer en el documento.

% Completar los datos de la caratula
\titulo{TPI - Agricultura con drones} 
\fecha{\today}
\materia{Algoritmos y Estructuras de Datos I}
\grupo{Grupo ?}

% Completar con cuantos integrantes quieran :)
\integrante{Apellido, Nombre1}{001/01}{email1@dominio.com}
\integrante{Apellido, Nombre2}{002/01}{email2@dominio.com}
\integrante{Apellido, Nombre3}{003/01}{email3@dominio.com}
\integrante{Apellido, Nombre4}{004/01}{email4@dominio.com}

\maketitle

\section{C\'odigo}
\lstinputlisting{codigo.cpp} % incluimos el codigo

\section{Demos}
    \subsection{M\'etodo: foo}
        \noindent
        \textbf{E0}\\
        vale: $|a| == n \land n > 0 \land a == pre(a)$ \\
        \textbf{E1}\\
         vale: $suma = 0 \land a == a@E0$ \\
        \textbf{E2}\\ 
        vale: $i == 0 \land suma == suma @E1 \land a == a@E1$ \\
        \textbf{E3}\\ 
        vale: $i == n \land suma == \sum a@E2 \land |a| == n \land  (\forall j \selec \rangoca{0}{n}) a_j == 0$\\
        \textbf{E4}\\ 
        vale: $res == suma@E3 / n \land suma == suma@E3 \land a == a@E3 \land i == i@E3$  \\

\end{document} %Termin�!
