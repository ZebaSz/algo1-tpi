\documentclass[a4paper]{article}

\usepackage[spanish]{babel} % Le indicamos a LaTeX que vamos a escribir en espa�ol.
\usepackage[latin1]{inputenc} % Permite utilizar tildes y e�es normalmente
\input{Algo1Macros}% Macros especificas para especificar problemas en AyEDI
\usepackage{caratula} % Se puede descargar en ~> https://github.com/bcardiff/dc-tex

\usepackage{listings} % Para poner codigo
\lstset{language=C++, breaklines=true, showstringspaces=false, mathescape=true, fontadjust, tabsize=4}

% Aca solo vamos a poner el esqueleto del documento, pero no vamos a especificar nada.

\begin{document} % Todo lo que escribamos a partir de aca va a aparecer en el documento.

% Completar los datos de la caratula
\titulo{TPI - Agricultura con drones} 
\fecha{\today}
\materia{Algoritmos y Estructuras de Datos I}
\grupo{Grupo 15}

% Completar con cuantos integrantes quieran :)
\integrante{Szperling, Sebastian}{763/15}{zebaszp@gmail.com}
\integrante{Barylko, Roni}{750/15}{ronibarylko@hotmail.com}
\integrante{Giudice, Carlos}{694/15}{Carlosr.giudice@gmail.com}
\integrante{Lopez Segura, Florencia}{759/13}{fsegura@dc.uba.ar}

\maketitle

\newcommand{\explicacion}[1]{\parbox{1\textwidth}{\hfill (#1)}}

\section{vueloEscalerado}

    \subsection{C\'odigo}
    \lstinputlisting{vueloEscalerado.cpp}

    \newpage

    \subsection{Demostraci\'on}
        \noindent
        \begin{Large}
        {$P_c \rightarrow I$}
        \end{Large} \\
        \\
        \textbf{Pc} \\
        vale $ P_c : i = 1 \land \longitud{vueloRealizado(this)} > 2 \land escalerado == (abs(dirx) == abs(diry)) \\ \land dirx == \prm{vueloRealizado(this)_0} - \prm{vueloRealizado(this)_2} \\ \land diry == \sgd{vueloRealizado(this)_0} - \sgd{vueloRealizado(this)_2} \\ $
        \\
		\textbf{I} \\
        $ I : 1 \leq i \leq \longitud{vueloRealizado(this)} -2 \ \land \ escalerado == (abs(dirx) == abs(diry) \ \land \ (\forall j \leftarrow [0..i)) \\ (\prm{vueloRealizado(this)_j} - \prm{vueloRealizado(this)_{j+2}} == dirx \\ \land \  \sgd{vueloRealizado(this)_j} - \sgd{vueloRealizado(this)_{j+2}} == diry) $ \\
        \\
        \textbf{Pc $\implica$ I} \\
        $ Pc \implica i == 1 \land \longitud{vueloRealizado(this)} > 2 \land escalerado == (abs(dirx) == abs(diry)) \\ \land dirx == \prm{vueloRealizado(this)_0} - \prm{vueloRealizado(this)_2} \\ \land diry == \sgd{vueloRealizado(this)_0} - \sgd{vueloRealizado(this)_2} \\
        \\
        \implica 1 \leq i \leq \longitud{vueloRealizado(this)} -2 \land escalerado == (abs(dirx) == abs(diry)) \\ \land dirx == \prm{vueloRealizado(this)_0} - \prm{vueloRealizado(this)_2} \\ \land diry == \sgd{vueloRealizado(this)_0} - \sgd{vueloRealizado(this)_2} \\
        \explicacion{vueloRealizado(this) - 2 $\geq$ 1} \\
        \implica 1 \leq i \leq \longitud{vueloRealizado(this)} -2 \ \land \ escalerado == (abs(dirx) == abs(diry) \\ \land \ (\prm{vueloRealizado(this)_0} - \prm{vueloRealizado(this)_2}) == dirx \\ \land \ (\sgd{vueloRealizado(this)_0} - \sgd{vueloRealizado(this)_2}) == diry) \\
        \explicacion{$true \land p == p$} \\
        \implica 1 \leq i \leq \longitud{vueloRealizado(this)} -2 \ \land \ escalerado == (abs(dirx) == abs(diry) \ \land \ (\forall j \leftarrow [0..i)) \\ (\prm{vueloRealizado(this)_j} - \prm{vueloRealizado(this)_{j+2}} == dirx \\ \land \ \sgd{vueloRealizado(this)_j} - \sgd{vueloRealizado(this)_{j+2}} == diry) \\
        \explicacion{dado $ i == 1$, $[0..i) == [0]$} \\
        \implica I $
        
        \newpage
        
        \begin{Large}
        {$(I \land \neg B) \implica Q_c$}
        \end{Large}\\
        \\
        \textbf{I}\\
        $ I : 1 \leq i \leq \longitud{vueloRealizado(this)} -2 \ \land \ escalerado == (abs(dirx) == abs(diry) \ \land \ (\forall j \leftarrow [0..i) (\prm{vueloRealizado(this)_i} - \prm{vueloRealizado(this)_{i+2}} == dirx \ \land \ \sgd{vueloRealizado(this)_i} - \sgd{vueloRealizado(this)_{i+2}} == diry) $ \\   
        \\
        \textbf{$\neg$B}\\
        $ \neg escalerado \ \lor \ i \geq \longitud{vueloRealizado(this)} - 2 $\\
        \\ 
        \textbf{Qc}\\ 
        vale $ Q_c : escalerado == (abs(dirx) == abs(diry) \ \land \ (\forall i \leftarrow [0..\longitud{vueloRealizado(this)} - 2)) \\ (\prm{vueloRealizado(this)_i} - \prm{vueloRealizado(this)_{i+2}} == dirx  \\ \land \ \sgd{vueloRealizado(this)_i} - \sgd{vueloRealizado(this)_{i+2}} == diry) $ \\ 
        \\
        \textbf{(I $\land \neg$ B) $\implica$ Qc} \\
        $I \land \neg B \\
        \implica (\neg escalerado \ \lor \ i \geq \longitud{vueloRealizado(this)} - 2) \land 1 \leq i \leq \longitud{vueloRealizado(this)} -2 \ \land \\ escalerado == (abs(dirx) == abs(diry) \ \land \ (\forall j \leftarrow [0..i)) (\prm{vueloRealizado(this)_i} - \prm{vueloRealizado(this)_{i+2}} \\ == dirx \ \land \ \sgd{vueloRealizado(this)_i} - \sgd{vueloRealizado(this)_{i+2}} == diry) \\
        \\
        \implica (\neg escalerado \land 1 \leq i \leq \longitud{vueloRealizado(this)} -2 \ \land \\ escalerado == (abs(dirx) == abs(diry) \land \ (\forall j \leftarrow [0..i)) (\prm{vueloRealizado(this)_i} - \prm{vueloRealizado(this)_{i+2}} \\ == dirx \ \land \ \sgd{vueloRealizado(this)_i} - \sgd{vueloRealizado(this)_{i+2}} == diry)) \\
        \lor % espacio para claridad; este o logico es gigante %
        (i == \longitud{vueloRealizado(this)} - 2 \land \\ escalerado == (abs(dirx) == abs(diry) \ \land \ (\forall j \leftarrow [0..i)) (\prm{vueloRealizado(this)_i} - \prm{vueloRealizado(this)_{i+2}} \\ == dirx \ \land \ \sgd{vueloRealizado(this)_i} - \sgd{vueloRealizado(this)_{i+2}} == diry)) $ \\
        \explicacion{$ (p \lor q) \land r \implica (p \land r) \lor (q \land r) $, y $ a \geq b \land a \leq b \implica a == b $} \\
        \\
        Podemos dividir esto en los dos casos contemplados por $\neg B$: \\
        \\
        Caso $\neg escalerado$ \\
        \\
        $ \neg escalerado \land 1 \leq i \leq \longitud{vueloRealizado(this)} -2 \ \land \\ escalerado == (abs(dirx) == abs(diry) \land \ (\forall j \leftarrow [0..i)) \\ (\prm{vueloRealizado(this)_i} - \prm{vueloRealizado(this)_{i+2}} \ == dirx \\ \land \ \sgd{vueloRealizado(this)_i} - \sgd{vueloRealizado(this)_{i+2}} == diry) \\
        \\
        \implica escalerado == (abs(dirx) == abs(diry) \ \land \ ((\forall j \leftarrow [0..i - 2)) \\ (\prm{vueloRealizado(this)_i} - \prm{vueloRealizado(this)_{i+2}} == dirx \\ \land \ \sgd{vueloRealizado(this)_i} - \sgd{vueloRealizado(this)_{i+2}} == diry)) \\ \land (((\forall j \leftarrow [i..\longitud{vueloRealizado(this)} - 2)) \\ (\prm{vueloRealizado(this)_i} - \prm{vueloRealizado(this)_{i+2}} == dirx \\ \land \ \sgd{vueloRealizado(this)_i} - \sgd{vueloRealizado(this)_{i+2}} == diry))) \\
        \explicacion{$\neg p \implica \neg(p \land q) $} \\
        \implica escalerado == (abs(dirx) == abs(diry) \ \land \ ((\forall j \leftarrow [0..\longitud{vueloRealizado(this)} - 2)) \\ (\prm{vueloRealizado(this)_i} - \prm{vueloRealizado(this)_{i+2}} == dirx \\ \land \ \sgd{vueloRealizado(this)_i} - \sgd{vueloRealizado(this)_{i+2}} == diry)) \\
        \\
        \implica Q_c \\
        \explicacion{por definicion} $\\
        \\
        \\
        Caso $i \geq \longitud{vueloRealizado(this)} - 2 $ \\
        \\
        $ i \geq \longitud{vueloRealizado(this)} - 2 \land 1 \leq i \leq \longitud{vueloRealizado(this)} -2 \ \land \\ escalerado == (abs(dirx) == abs(diry) \land \ (\forall j \leftarrow [0..i)) \\ (\prm{vueloRealizado(this)_i} - \prm{vueloRealizado(this)_{i+2}} \ == dirx \\ \land \ \sgd{vueloRealizado(this)_i} - \sgd{vueloRealizado(this)_{i+2}} == diry) \\
        \\
        \implica i == \longitud{vueloRealizado(this)} - 2 \land \\ escalerado == (abs(dirx) == abs(diry) \land \ (\forall j \leftarrow [0..i)) \\ (\prm{vueloRealizado(this)_i} - \prm{vueloRealizado(this)_{i+2}} \ == dirx \\ \land \ \sgd{vueloRealizado(this)_i} - \sgd{vueloRealizado(this)_{i+2}} == diry) \\
        \explicacion{$a \leq b \land b \leq a \implica a == b$} \\
        \implica escalerado == (abs(dirx) == abs(diry) \land \ (\forall j \leftarrow [0..\longitud{vueloRealizado(this)} - 2)) \\ (\prm{vueloRealizado(this)_i} - \prm{vueloRealizado(this)_{i+2}} \ == dirx \\ \land \ \sgd{vueloRealizado(this)_i} - \sgd{vueloRealizado(this)_{i+2}} == diry) \\
        \\
        \implica Q_c \\
        \explicacion{por definicion} $ \\
        \\
        Considerando ambos casos: \\
        $ Q_c \land Q_c \implica Q_c $ \\
        \explicacion{Q.E.D.}

        \newpage

        \begin{Large}
        {$(I \land fv \leq cota) \implica \neg B$}
        \end{Large}\\
        \\
        \textbf{I}\\
        $ I : 1 \leq i \leq \longitud{vueloRealizado(this)} -2 \ \land \ escalerado == (abs(dirx) == abs(diry) \ \land \ (\forall j \leftarrow [0..i) \\ (\prm{vueloRealizado(this)_i} - \prm{vueloRealizado(this)_{i+2}} == dirx \\ \land \ \sgd{vueloRealizado(this)_i} - \sgd{vueloRealizado(this)_{i+2}} == diry) $ \\ 
        \\
        \textbf{fv}\\
        $ f_v: \longitud{vueloRealizado(this)} - i - 2 $\\
        Cota: $0$\\
        \\
        \textbf{$\neg$B}\\
        $\neg escalerado \ \lor \ i \geq \longitud{vueloRealizado(this)} - 2$ \\
        \\
        \textbf{I $\land$ fv $\leq$ cota $\implica \neg$ B}\\
        \\ $I \land f_v \leq cota \\
        \implica \longitud{vueloRealizado(this)} - i - 2 \leq 0 \ \land \ 1 \leq i \leq \longitud{vueloRealizado(this)} -2 \\ \land \ escalerado == (abs(dirx) == abs(diry) \ \land \ (\forall j \leftarrow [0..i) \\ (\prm{vueloRealizado(this)_i} - \prm{vueloRealizado(this)_{i+2}} == dirx \\ \land \ \sgd{vueloRealizado(this)_i} - \sgd{vueloRealizado(this)_{i+2}} == diry) \\
        \explicacion{por definicion} \\
        \implica \longitud{vueloRealizado(this)} - i - 2 \leq 0 \\
        \explicacion{$p \land q \implica p$}\\
        \implica i \geq \longitud{vueloRealizado(this)} - 2 \\
        \explicacion{por inecuaciones}\\
        \implica \neg escalerado \ \lor \ i \geq \longitud{vueloRealizado(this)} - 2 \\
        \explicacion{$p \implica p \lor q$} \\
        \implica \neg B \\
        \explicacion{por definicion} \\
        $ \\
\newpage

\section{listoParaCosechar}

    \subsection{C\'odigo}
    \lstinputlisting{listoParaCosechar.cpp}

    \subsection{Demostraci\'on}
        \noindent
        \begin{large}
       {$P_c \rightarrow I$}
        \end{large} \\
        \\
        \textbf{Pc} \\
        vale $ P_c : cantCosechables == 0 \land i == 0$ \\
        \\
		\textbf{I} \\
        $ I: 0 \leq i \leq (prm(dimensiones(campo(this)) \times sgd(dimensiones(campo(this)))) $\\ $ \land | [1| x \leftarrow [0..i \div sgd(dimensiones(campo(this)))) $\\ $ y \leftarrow [0.. (i-1) \ mod \ sgd(dimensiones(campo(this)))], estadoDelCultivo((x,y), this) == ListoParaCosechar]| $\\$ == cantCosechables $\\   
\\ $Pc \implica$\\
$i == 0 \implica$\\
$ 0 \leq i \leq (prm(dimensiones(campo(this)) \times sgd(dimensiones(campo(this)))) \land $\\$
[1| x \leftarrow [0..0) $\\ $ y \leftarrow [0.. sgd(dimensiones(campo(this) - 1], estadoDelCultivo((x,y), this) == ListoParaCosechar]| == cantCosechables$\\
\\ Pc asegura i==0 $ \land $ cantCosechables == 0, lo cual implica que $ [1| x \leftarrow [0..0) $\\ $ y \leftarrow [0.. sgd(dimensiones(campo(this) - 1], estadoDelCultivo((x,y), this) == ListoParaCosechar]|$ no recorra ninguna posicion ( $x \leftarrow \lvacia$) y por lo tanto longitud == 0 == cantCosechables\\
Entonces $P_c \implica I$

\newpage
		\begin{Large}
        {$(I \land \neg B) \implica Qc$}
        \end{Large}\\
        \\
        \textbf{I}\\
         $ I: 0 \leq i \leq (prm(dimensiones(campo(this)) \times sgd(dimensiones(campo(this)))) $\\ 	$ \land | [1| x \leftarrow [0..i \div sgd(dimensiones(campo(this)))) $\\ $ y \leftarrow 	[0.. (i-1) \ mod \ sgd(dimensiones(campo(this)))], estadoDelCultivo((x,y), this) == ListoParaCosechar]| $\\$ == cantCosechables $ \\   
        \\
        \textbf{$\neg$B}\\
        $(prm(dimensiones(campo(this)) \times sgd(dimensiones(campo(this)))) \leq i$\\
        \\ 
        \textbf{Qc}\\
         vale $Q_c$:$i == (prm(dimensiones(campo(this)) \times sgd(dimensiones(campo(this)))) \land $\\ $ \longitud{[1| pos \leftarrow parcelasCultivo(campo(this)), estadoDelCultivo(pos, this) == ListoParaCosechar]} $\\ $ == cantCosechables $
         
$(I \land \neg B) \implica$\\
$i == (prm(dimensiones(campo(this)) \times sgd(dimensiones(campo(this)))) $\\$ \land | [1| x \leftarrow [0..i \div sgd(dimensiones(campo(this)))) $\\ $ y \leftarrow 	[0.. i \ mod \ sgd(dimensiones(campo(this)))), estadoDelCultivo((x,y), this) == ListoParaCosechar]| $\\$ == cantCosechables \implica$\\
$i == (prm(dimensiones(campo(this)) \times sgd(dimensiones(campo(this)))) $\\$ \land | [1| x \leftarrow [0..prm(dimensiones(campo(this)))) $\\ $ y \leftarrow 	[0..sgd(dimensiones(campo(this))) - 1], estadoDelCultivo((x,y), this) == ListoParaCosechar]| $\\$ == cantCosechables \implica$\\
$ i == (prm(dimensiones(campo(this)) \times sgd(dimensiones(campo(this)))) $\\$ \land | [1| x \leftarrow [0..prm(dimensiones(campo(this)))) $\\ $ y \leftarrow 	[0..sgd(dimensiones(campo(this)))), estadoDelCultivo((x,y), this) == ListoParaCosechar]| $\\$ == cantCosechables $\\

$\neg B \ \land \ I$  asegura que $ i == (prm(dimensiones(campo(this)) \times sgd(dimensiones(campo(this))))$ y que cantCosechables es igual a la longitud de la lista correspondiente, pero al ser  $i == (prm(dimensiones(campo(this)) \times sgd(dimensiones(campo(this))))$ esto nos permite asegurar que las variables x e y recorren todo el campo. De este modo, tenemos el invariante extendido para todo el campo y la cantCosechables es igual a la longitud de la lista, lo que es exactamente Qc. \\
Por lo tanto,$(I \land \neg B) \implica Qc$\\
		\\
        \begin{Large}
        {$(I \land fv < cota) \implica \neg B$}
        \end{Large}\\
        \\
        \textbf{I}\\
         $ I: 0 \leq i \leq (prm(dimensiones(campo(this)) \times sgd(dimensiones(campo(this)))) $\\ 	$ \land | [1| x \leftarrow [0..i \div sgd(dimensiones(campo(this)))) $\\ $ y \leftarrow 	[0.. (i-1) \ mod \ sgd(dimensiones(campo(this)))], estadoDelCultivo((x,y), this) == ListoParaCosechar]| $\\$ == cantCosechables $ \\
        \\
        \textbf{fv}\\
        Funcion variante fv: $ prm(dimensiones(campo(this)) \times sgd(dimensiones(campo(this))) - i $\\
        Cota: $1$\\
        \\
       \textbf{$\neg$B}\\
        $(prm(dimensiones(campo(this)) \times sgd(dimensiones(campo(this)))) \leq i$\\
        \\
$(I \land fv < cota) \implica$\\
$prm(dimensiones(campo(this)) \times sgd(dimensiones(campo(this))) - i < 1 \implica$\\
$prm(dimensiones(campo(this)) \times sgd(dimensiones(campo(this))) - i \leq 0 \implica$\\
$prm(dimensiones(campo(this)) \times sgd(dimensiones(campo(this))) \leq i \ \land \ $\\$ prm(dimensiones(campo(this)) \times sgd(dimensiones(campo(this))) \geq i \implica$\\
$i == prm(dimensiones(campo(this)) \times sgd(dimensiones(campo(this)))$\\
\\
$fv < cota$ implica que i sea al menos $prm(dimensiones(campo(this)) \times sgd(dimensiones(campo(this)))$ y el invariante, que sea menor a esa expresion. Por lo tanto, i es igual a la expresion, lo cual implica la negacion de la guarda ($\neg ((prm(dimensiones(campo(this)) \times sgd(dimensiones(campo(this)))) > i))$
Por lo tanto, $(I \land \neg B) \implica Qc$
\end{document} %Termin�!
