\documentclass[a4paper]{article}

\usepackage[spanish]{babel} % Le indicamos a LaTeX que vamos a escribir en espa�ol.
\usepackage[latin1]{inputenc} % Permite utilizar tildes y e�es normalmente
\input{Algo1Macros}% Macros especificas para especificar problemas en AyEDI
\usepackage{caratula} % Se puede descargar en ~> https://github.com/bcardiff/dc-tex

\usepackage{listings} % Para poner codigo
\lstset{language=C++, breaklines=true, showstringspaces=false, mathescape=true, fontadjust, tabsize=4}

% Aca solo vamos a poner el esqueleto del documento, pero no vamos a especificar nada.

\begin{document} % Todo lo que escribamos a partir de aca va a aparecer en el documento.

% Completar los datos de la caratula
\titulo{TPI - Agricultura con drones} 
\fecha{\today}
\materia{Algoritmos y Estructuras de Datos I}
\grupo{Grupo 15}

% Completar con cuantos integrantes quieran :)
\integrante{Szperling, Sebastian}{763/15}{zebaszp@gmail.com}
\integrante{Barylko, Roni}{750/15}{ronibarylko@hotmail.com}
\integrante{Giudice, Carlos}{694/15}{Carlosr.giudice@gmail.com}
\integrante{Lopez Segura, Florencia}{759/13}{fsegura@dc.uba.ar}

\maketitle

\section{C\'odigo}
\lstinputlisting{codigo.cpp} % incluimos el codigo

\section{Demos}
    \subsection{M\'etodo: vueloEscalerado}
        \noindent
        \begin{Large}
        {$Pc \rightarrow I$}
        \end{Large}\\
        \\
        \textbf{Pc}\\
        vale $ P_c $: $ i = 0 \land \longitud{vueloRealizado(this)} > 2 \land escalerado == (enVuelo(this) \land \mid dirx \mid == 1 \ \land \mid diry \mid == 1) $ \\
        \\
		\textbf{I}\\
        invariante:$ 0 \leq i \leq \longitud{vueloRealizado(this)} -2 \land \ escalerado == ((\forall j \leftarrow [0..i -1)) $ $ (\prm{vueloRealizado(this)_i} - \prm{vueloRealizado(this)_{i+2}} == dirx $ $ \land \ \sgd{vueloRealizado(this)_i} - \sgd{vueloRealizado(this)_{i+2}} == diry) $ \\   
\\ $Pc \implica 0 \leq i \leq \longitud{vueloRealizado(this)} -2 \land [0..i-1] == []$ \\
\\ Por lo tanto, Pc implica ambas condiciones del invariante (i pertenece al rango, y la condicion escalerado == enVuelo $\land$ $\forall j \leftarrow []$ devuelve enVuelo) \\ entonces $Pc \implica I$ \\
\\
\begin{Large}
        {$I \land \neg B \implica Qc$}
        \end{Large}\\
        \\
        \textbf{I}\\
        invariante: $ 0 \leq i \leq \longitud{vueloRealizado(this)} -2 \land escalerado == ((\forall j \leftarrow [0..i -1)) $ $ (\prm{vueloRealizado(this)_i} - \prm{vueloRealizado(this)_{i+2}} == dirx $ $ \land \ \sgd{vueloRealizado(this)_i} - \sgd{vueloRealizado(this)_{i+2}} == diry) $ \\
        \\
        \textbf{$\neg$B}\\
        $\neg$escalerado$ \ \vee \ i \geq \longitud{vueloRealizado(this)} - 2$
       
\end{document} %Termin�!
