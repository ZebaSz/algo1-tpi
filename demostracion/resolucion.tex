\documentclass[a4paper]{article}

\usepackage[spanish]{babel} % Le indicamos a LaTeX que vamos a escribir en espa�ol.
\usepackage[latin1]{inputenc} % Permite utilizar tildes y e�es normalmente
\input{Algo1Macros}% Macros especificas para especificar problemas en AyEDI
\usepackage{caratula} % Se puede descargar en ~> https://github.com/bcardiff/dc-tex

\usepackage{bm}

\usepackage{listings} % Para poner codigo
\lstset{language=C++, breaklines=true, showstringspaces=false, mathescape=true, fontadjust, tabsize=4}

% Aca solo vamos a poner el esqueleto del documento, pero no vamos a especificar nada.

\begin{document} % Todo lo que escribamos a partir de aca va a aparecer en el documento.

% Completar los datos de la caratula
\titulo{TPI - Agricultura con drones} 
\fecha{\today}
\materia{Algoritmos y Estructuras de Datos I}
\grupo{Grupo 15}

% Completar con cuantos integrantes quieran :)
\integrante{Szperling, Sebastian}{763/15}{zebaszp@gmail.com}
\integrante{Barylko, Roni}{750/15}{ronibarylko@hotmail.com}
\integrante{Giudice, Carlos}{694/15}{Carlosr.giudice@gmail.com}
\integrante{Lopez Segura, Florencia}{759/13}{fsegura@dc.uba.ar}

\maketitle

\setlength{\parindent}{0cm}

\newcommand{\explicacion}[1]{\parbox{1\textwidth}{\hfill (#1)}}

\section{vueloEscalerado}

    \subsection{C\'odigo}
        \lstinputlisting{vueloEscalerado.cpp}

    \newpage

    \subsection{Demostraci\'on}
        
        \begin{Large}
        {$P_c \rightarrow I$}
        \end{Large}

        \bigskip
        \textbf{Pc}

        $ i == 1 \land \longitud{vueloRealizado(this)} > 2 \land escalerado == (dirx \neq 0 \land diry \neq 0) \\ \land dirx == \prm{vueloRealizado(this)_0} - \prm{vueloRealizado(this)_2} \\ \land diry == \sgd{vueloRealizado(this)_0} - \sgd{vueloRealizado(this)_2} $

        \bigskip
		\textbf{I}

        $ 1 \leq i \leq \longitud{vueloRealizado(this)} -2 \ \land \ escalerado == (dirx \neq 0 \land diry \neq 0 \ \land \ (\forall j \leftarrow [0..i)) \\ (\prm{vueloRealizado(this)_j} - \prm{vueloRealizado(this)_{j+2}} == dirx \\ \land \  \sgd{vueloRealizado(this)_j} - \sgd{vueloRealizado(this)_{j+2}} == diry) $

        \bigskip
        \textbf{Pc $\implica$ I}

        Por Pc, sabemos que i == 1 y $|vueloRealizado(this)| > 2$. Por lo tanto, $1 \leq i$ (porque que sea igual, implica que sea igual o mayor) y $i \leq 2 \leq |vueloRealizado(this)|$ (porque que sea mayor, implica que sea igual o mayor).

        Luego, $1 \leq i \leq |vueloRealizado(this)|$

        \bigskip
        Por otro lado, por Pc sabemos que $dirx == \prm{vueloRealizado(this)_0} - \prm{vueloRealizado(this)_2} \\ \land diry == \sgd{vueloRealizado(this)_0} - \sgd{vueloRealizado(this)_2}$ y que $escalerado == (dirx \neq 0 \land diry \neq 0)$.

        Por lo tanto, podemos afirmar que
        
        \bigskip
        $escalerado == (dirx \neq 0 \land diry \neq 0 \land dirx == \prm{vueloRealizado(this)_0} - \prm{vueloRealizado(this)_2} \\ \land diry == \sgd{vueloRealizado(this)_0} - \sgd{vueloRealizado(this)_2})$
        
        (como la segunda parte es $true$, no modifica el valor de escalerado)

        \bigskip
        Finalmente, como i == 1 por Pc, podemos tomar la lista $[0..i) == [0..1) == [0]$. Por lo tanto
        
        \bigskip
        $(\forall j \leftarrow [0..i)) (\prm{vueloRealizado(this)_j} - \prm{vueloRealizado(this)_{j+2}} == dirx \\ \land \ \sgd{vueloRealizado(this)_j} - \sgd{vueloRealizado(this)_{j+2}} == diry)$

        \bigskip
        es equivalente a $dirx == \prm{vueloRealizado(this)_0} - \prm{vueloRealizado(this)_2} \\ \land diry == \sgd{vueloRealizado(this)_0} - \sgd{vueloRealizado(this)_2}$. Es decir

        \bigskip
        $escalerado == (dirx \neq 0 \land diry \neq 0 \ \land \ (\forall j \leftarrow [0..i)) \\ (\prm{vueloRealizado(this)_j} - \prm{vueloRealizado(this)_{j+2}} == dirx \\ \land \ \sgd{vueloRealizado(this)_j} - \sgd{vueloRealizado(this)_{j+2}} == diry)$

        \bigskip
        Luego, con estos valores de $i$ y de $escalerado$ queda exactamente el invariante. Por lo tanto, Pc implica I.
        
        \newpage
        
        \begin{Large}
        {$(I \land \neg B) \implica Q_c$}
        \end{Large}

        \bigskip
        \textbf{I}

        $ 1 \leq i \leq \longitud{vueloRealizado(this)} -2 \ \land \ escalerado == (dirx \neq 0 \land diry \neq 0 \ \land \ (\forall j \leftarrow [0..i) (\prm{vueloRealizado(this)_j} - \prm{vueloRealizado(this)_{j+2}} == dirx \ \land \ \sgd{vueloRealizado(this)_j} - \sgd{vueloRealizado(this)_{j+2}} == diry) $   

        \bigskip
        \textbf{$\neg$B}

        $ \neg escalerado \ \lor \ i \geq \longitud{vueloRealizado(this)} - 2 $

        \bigskip
        \textbf{Qc}

        $ escalerado == (dirx \neq 0 \land diry \neq 0 \ \land \ (\forall j \leftarrow [0..\longitud{vueloRealizado(this)} - 2)) \\ (\prm{vueloRealizado(this)_j} - \prm{vueloRealizado(this)_{j+2}} == dirx  \\ \land \ \sgd{vueloRealizado(this)_j} - \sgd{vueloRealizado(this)_{j+2}} == diry) $

        \bigskip
        \textbf{(I $\land \neg$ B) $\implica$ Qc}

        Por $\neg$B, sabemos que $ \neg escalerado \ \lor \ i \geq \longitud{vueloRealizado(this)} - 2 $. Por reglas l\'ogicas, basta con que ocurra una de las dos cosas para que se cumpla $\neg$B. Por lo tanto, podemos abrir el problema en dos casos.

        \bigskip
        \underline{Caso $\neg escalerado$}

        \bigskip
        Por I, $escalerado == (dirx \neq 0 \land diry \neq 0 \ \land \ (\forall j \leftarrow [0..i) (\prm{vueloRealizado(this)_j} - \prm{vueloRealizado(this)_{j+2}} \\ == dirx \ \land \ \sgd{vueloRealizado(this)_j} - \sgd{vueloRealizado(this)_{j+2}} == diry) $ y por $\neg$B, $\neg$escalerado. Luego,

        \bigskip
        $\neg(dirx \neq 0 \land diry \neq 0 \ \land \ (\forall j \leftarrow [0..i) (\prm{vueloRealizado(this)_j} - \prm{vueloRealizado(this)_{j+2}} == dirx \\ \land \ \sgd{vueloRealizado(this)_j} - \sgd{vueloRealizado(this)_{j+2}} == diry)$
        
        \bigskip
        Si llamamos $ losDemas : (((\forall j \leftarrow [i..\longitud{vueloRealizado(this)} - 2)$\\ $(\prm{vueloRealizado(this)_j} - \prm{vueloRealizado(this)_{j+2}} == dirx \\ \land \ \sgd{vueloRealizado(this)_j} - \sgd{vueloRealizado(this)_{j+2}} == diry)))$
        
        \bigskip
        como vale $\neg$escalerado, $losDemas \land escalerado == false$.

        \bigskip
        Luego, $escalerado == (dirx \neq 0 \land diry \neq 0 \ \land \ ((\forall j \leftarrow [0..i)) \\ (\prm{vueloRealizado(this)_j} - \prm{vueloRealizado(this)_{j+2}} == dirx \\ \land \ \sgd{vueloRealizado(this)_j} - \sgd{vueloRealizado(this)_{j+2}} == diry)) \\ \land (((\forall j \leftarrow [i..\longitud{vueloRealizado(this)} - 2))(\prm{vueloRealizado(this)_j} - \prm{vueloRealizado(this)_{j+2}} == dirx \\ \land \ \sgd{vueloRealizado(this)_j} - \sgd{vueloRealizado(this)_{j+2}} == diry))$

        \bigskip
        Por otro lado, como aplicamos la misma condicion para $((\forall j \leftarrow [0..i))$ y $(((\forall j \leftarrow [i..\longitud{vueloRealizado(this)} - 2))$, podemos concatenar ambas listas.

        \bigskip
        Luego, $ escalerado == (dirx \neq 0 \land diry \neq 0 \ \land \ (\forall j \leftarrow [0..\longitud{vueloRealizado(this)} - 2)) \\ (\prm{vueloRealizado(this)_j} - \prm{vueloRealizado(this)_{j+2}} == dirx  \\ \land \ \sgd{vueloRealizado(this)_j} - \sgd{vueloRealizado(this)_{j+2}} == diry) $

        \bigskip
        De este modo, el caso $\neg escalerado \land I$ implica Qc

        \bigskip
        \underline{Caso $\ i \geq \longitud{vueloRealizado(this)} - 2$}

        \bigskip
        Por I,  $ 1 \leq i \leq \longitud{vueloRealizado(this)} -2$ y por $\neg$B, $i \geq \longitud{vueloRealizado(this)} - 2 $.
        
        Luego, i == $\longitud{vueloRealizado(this)} - 2$.

        \bigskip
        Por lo tanto, la condicion $(\forall j \leftarrow [0..i)$ es igual a $(\forall j \leftarrow [0..\longitud{vueloRealizado(this)} - 2))$

        \bigskip
        Luego, $ escalerado == (dirx \neq 0 \land diry \neq 0 \ \land \ (\forall j \leftarrow [0..\longitud{vueloRealizado(this)} - 2)) \\ (\prm{vueloRealizado(this)_j} - \prm{vueloRealizado(this)_{j+2}} == dirx  \\ \land \ \sgd{vueloRealizado(this)_j} - \sgd{vueloRealizado(this)_{j+2}} == diry) $

        \bigskip
        De este modo, el caso $\ i \geq \longitud{vueloRealizado(this)} - 2 \land I$ tambi\'en implica Qc.

        \bigskip
        Finalmente, al unir ambos casos, tenemos Qc $\lor$ Qc. Es decir, que ambos caminos permiten implicar Qc.

        Por lo tanto I y $\neg$B implica Qc.

        \newpage

        \begin{Large}
        {$(I \land fv \leq cota) \implica \neg B$}
        \end{Large}

        \bigskip
        \textbf{I}

        $ 1 \leq i \leq \longitud{vueloRealizado(this)} -2 \ \land \ escalerado == (dirx \neq 0 \land diry \neq 0 \ \land \ (\forall j \leftarrow [0..i) \\ (\prm{vueloRealizado(this)_j} - \prm{vueloRealizado(this)_{j+2}} == dirx \\ \land \ \sgd{vueloRealizado(this)_j} - \sgd{vueloRealizado(this)_{j+2}} == diry) $

        \bigskip
        \textbf{fv}

        $ \longitud{vueloRealizado(this)} - i - 2 $\\
        Cota: $0$

        \bigskip
        \textbf{$\neg$B}

        $\neg escalerado \ \lor \ i \geq \longitud{vueloRealizado(this)} - 2$

        \bigskip
        \textbf{I $\land$ fv $\leq$ cota $\implica \neg$ B}

        \bigskip
        Por $fv \leq cota$, sabemos que $\longitud{vueloRealizado(this)} - i - 2 \leq 0$.

        \bigskip
        Luego, sumando i de ambos lados, obtenemos $\longitud{vueloRealizado(this)} - 2 {\leq i}$

        \bigskip
        Del mismo modo, por reglas logicas $\longitud{vueloRealizado(this)} - 2 {\leq i}$ implica $\longitud{vueloRealizado(this)} - 2 {\leq i} \lor q$ (porque basta que se cumpla una de las condiciones para que el $\lor$ devuelva $true$).

        \bigskip
        Luego, $\longitud{vueloRealizado(this)} - 2 {\leq i} \lor \neg$escalerado, que es exactamente $\neg$B.

        \bigskip
        De este modo, demostramos que $(I \land fv \leq cota)$ implica $\neg$B

\newpage

\section{listoParaCosechar}

    \subsection{C\'odigo}
        \lstinputlisting{listoParaCosechar.cpp}

        \bigskip
        Para estas demostraciones, se utiliza la siguiente funci\'on auxiliar:

        \bigskip
        \aux{min}{a, b: \ent}{\ent}{\IfThenElse{a < b}{a}{b}}

    \newpage

    \subsection{Demostraci\'on}
        
        \begin{large}
        {$P_c \rightarrow I$}
        \end{large}

        \bigskip
        \textbf{Pc}

        $ cantCosechables == 0 \land i == 0$

        \bigskip
		\textbf{I}

        $ 0 \leq i \leq (\prm{dimensiones(campo(this))} \times \sgd{dimensiones(campo(this))}) \\ \land | [1| x \leftarrow [0..i \div \sgd{dimensiones(campo(this))}], \\ y \leftarrow [0..min(\sgd{dimensiones(campo(this))}, i - x * \sgd{dimensiones(campo(this))})), \\ estadoDelCultivo((x,y), this) == ListoParaCosechar]| == cantCosechables $

        \bigskip
        \textbf{Pc $\implica$ I}

        Por Pc, sabemos que $i == 0$. Por lo tanto, $0 \leq i$ (porque que sea igual, implica que sea igual o mayor) y \\ $i \leq (\prm{dimensiones(campo(this))} \times \sgd{dimensiones(campo(this))})$ (porque que sea mayor, implica que sea igual o mayor).

        Luego, $0 \leq i \leq (\prm{dimensiones(campo(this))} \times \sgd{dimensiones(campo(this))})$.

        \bigskip
        Por otro lado, sabiendo que $i == 0$, podemos tomar la lista [0..i] como la lista [0], y la lista [0..0) como la lista vac\'ia. De esta manera, podemos formar la lista

        \bigskip
        $[1| x \leftarrow [0..i \div \sgd{dimensiones(campo(this))}], \\ y \leftarrow [0..min(\sgd{dimensiones(campo(this))}, i - x * \sgd{dimensiones(campo(this))})), \\ estadoDelCultivo((x,y), this) == ListoParaCosechar]$

        \bigskip
        de longitud 0 (la lista est\'a vac\'ia ya que $y$ no toma ning\'un valor). Por Pc, $cantCosechables == 0$, que es equivalente a la longitud de esta lista, por lo que podemos afirmar

        \bigskip
        $cantCosechables == |[1| x \leftarrow [0..i \div \sgd{dimensiones(campo(this))}], \\ y \leftarrow [0..min(\sgd{dimensiones(campo(this))}, i - x * \sgd{dimensiones(campo(this))})), \\ estadoDelCultivo((x,y), this) == ListoParaCosechar]|$

        \bigskip
        Tomando ambos datos, afirmamos que

        \bigskip
        $ 0 \leq i \leq (\prm{dimensiones(campo(this))} \times \sgd{dimensiones(campo(this))}) \ \land \ cantCosechables == \\ |[1| x \leftarrow [0..i \div \sgd{dimensiones(campo(this))}], \\ y \leftarrow [0..min(\sgd{dimensiones(campo(this))}, i - x * \sgd{dimensiones(campo(this))})), \\ estadoDelCultivo((x,y), this) == ListoParaCosechar]| $

        \bigskip
        que es equivalente a I. De esta manera, demostramos que Pc $\implica$ I.

        \newpage

		\begin{Large}
        {$(I \land \neg B) \implica Q_c$}
        \end{Large}

        \bigskip
        \textbf{I}

        $ 0 \leq i \leq (\prm{dimensiones(campo(this))} \times \sgd{dimensiones(campo(this))}) \\ \land | [1| x \leftarrow [0..i \div \sgd{dimensiones(campo(this))}], \\ y \leftarrow [0..min(\sgd{dimensiones(campo(this))}, i - x * \sgd{dimensiones(campo(this))})), \\ estadoDelCultivo((x,y), this) == ListoParaCosechar]| == cantCosechables $

        \bigskip
        \textbf{$\neg$B}

        $ (\prm{dimensiones(campo(this))} \times \sgd{dimensiones(campo(this))}) \leq i $

        \bigskip
        \textbf{Qc}

        $ i == (\prm{dimensiones(campo(this))} \times \sgd{dimensiones(campo(this))}) \ \land cantCosechables == \\ | [1| x \leftarrow [0..\prm{dimensiones(campo(this))}), y \leftarrow [0..\sgd{dimensiones(campo(this))}), \\ estadoDelCultivo((x,y), this) == ListoParaCosechar] | $

        \bigskip
        \textbf{(I $\land \neg$ B) $\implica$ Qc}

        Por I, sabemos que $0 \leq i \leq (\prm{dimensiones(campo(this))} \times \sgd{dimensiones(campo(this))})$. Adem\'as, por $\neg$B, sabemos que $(\prm{dimensiones(campo(this))} \times \sgd{dimensiones(campo(this))}) \leq i$.

        De esta conjunci\'on deducimos que $ i == (\prm{dimensiones(campo(this))} \times \sgd{dimensiones(campo(this))})$ (si un n\'umero es menor o igual y mayor o igual que otro, entonces son iguales).

        \bigskip
        Teniendo esto, y sabiendo por I que

        \bigskip
        $ cantCosechables == | [1| x \leftarrow [0..i \div \sgd{dimensiones(campo(this))}], \\ y \leftarrow [0..min(\sgd{dimensiones(campo(this))}, i - x * \sgd{dimensiones(campo(this))})), \\ estadoDelCultivo((x,y), this) == ListoParaCosechar]| $

        \bigskip
        podemos modificar las listas de los selectores. Por un lado

        \bigskip
        $i \div \sgd{dimensiones(campo(this))} == \prm{dimensiones(campo(this))}$

        \bigskip
        Por lo tanto, $x \leftarrow [0..\prm{dimensiones(campo(this))}]$. Por otro lado, considerando nuevamente el valor de i y sacando factor com\'un $\sgd{dimensiones(campo(this))}$

        \bigskip
        $i - x * \sgd{dimensiones(campo(this))} == (\prm{dimensiones(campo(this))} - x) * \sgd{dimensiones(campo(this))}$

        \bigskip
        Por lo tanto,

        $y \leftarrow [0..min(\sgd{dimensiones(campo(this))}, (\prm{dimensiones(campo(this))} - x) * \sgd{dimensiones(campo(this))})$.

        Ahora, podemos dividir el min en dos casos

        \bigskip
        \underline{Caso $x == \prm{dimensiones(campo(this))}$}

        \bigskip
        Como $i - x * \sgd{dimensiones(campo(this))}) == (\prm{dimensiones(campo(this))} - x) * \sgd{dimensiones(campo(this))})$
        
        \bigskip
        $(\prm{dimensiones(campo(this))} - x) * \sgd{dimensiones(campo(this))}) == 0 $
        
        \bigskip
        y dado que $sgd(dimensiones(campo(this))$ es positivo, en este caso el $min$ devuelve 0.

        \bigskip
        \underline{Caso $x \neq \prm{dimensiones(campo(this))}$}

        \bigskip
        Dado $x \leftarrow [0..\prm{dimensiones(campo(this))}]$ y $x \neq \prm{dimensiones(campo(this))}$, \\ se toman valores de la lista $[0..\prm{dimensiones(campo(this))})$ (sin incluir el extremo).

        Como $(\prm{dimensiones(campo(this))} - x)$ es natural y positivo, deducimos que

        \bigskip
        $\sgd{dimensiones(campo(this))} \leq (\prm{dimensiones(campo(this))} - x) * \sgd{dimensiones(campo(this))}) $

        \bigskip
        Uniendo ambos casos, podemos expresar cantCosechables como

        \bigskip
        $ cantCosechables == | [1| x \leftarrow [0..\prm{dimensiones(campo(this))}), y \leftarrow [0..\sgd{dimensiones(campo(this))}), \\ estadoDelCultivo((x,y), this) == ListoParaCosechar] \\ ++ [ 1| x \leftarrow [\prm{dimensiones(campo(this))}], y \leftarrow [0..$$0), estadoDelCultivo((x,y), this) == ListoParaCosechar] | $

        \bigskip
        Sin embargo, [0..0) resulta en la lista vac\'ia, por lo que la segunda lista de esta concatenaci\'on no tiene elementos y podemos quitarla. 

        \bigskip
        $ cantCosechables == | [1| x \leftarrow [0..\prm{dimensiones(campo(this))}), y \leftarrow [0..\sgd{dimensiones(campo(this))}), \\ estadoDelCultivo((x,y), this) == ListoParaCosechar] | $

        \bigskip
        Considerando los valores de i y cantCosechables, tenemos que

        \bigskip
        $ i == (\prm{dimensiones(campo(this))} \times \sgd{dimensiones(campo(this))}) \land cantCosechables == \\ | [1| x \leftarrow [0..\prm{dimensiones(campo(this))}), y \leftarrow [0..\sgd{dimensiones(campo(this))}), \\ estadoDelCultivo((x,y), this) == ListoParaCosechar] | $

        \bigskip
        que por definici\'on es equivalente a Qc. Por lo tanto I y $\neg$B implica Qc.

        \newpage

        \begin{Large}
        {$(I \land fv \leq cota) \implica \neg B$}
        \end{Large}

        \bigskip
        \textbf{I}

        $ 0 \leq i \leq (\prm{dimensiones(campo(this))} \times \sgd{dimensiones(campo(this))}) \\ \land | [1| x \leftarrow [0..i \div \sgd{dimensiones(campo(this))}], \\ y \leftarrow [0..min(\sgd{dimensiones(campo(this))}, i - x * \sgd{dimensiones(campo(this))})), \\ estadoDelCultivo((x,y), this) == ListoParaCosechar]| == cantCosechables $

        \bigskip
        \textbf{fv}

        $ \prm{dimensiones(campo(this))} \times \sgd{dimensiones(campo(this))} - i $\\
        Cota: $0$

        \bigskip
        \textbf{$\neg$B}

        $(\prm{dimensiones(campo(this))} \times \sgd{dimensiones(campo(this))}) \leq i$

        \bigskip
        \textbf{I $\land$ fv $\leq$ cota $\implica \neg$B}

        \bigskip
        Por $fv \leq cota$, sabemos que $\prm{dimensiones(campo(this))} \times \sgd{dimensiones(campo(this))} - i \leq 0 $.

        \bigskip
        Luego, sumando i de ambos lados, obtenemos 
        
        \bigskip
        $\prm{dimensiones(campo(this))} \times \sgd{dimensiones(campo(this))} \leq i$
        
        \bigskip
        que es exactamente $\neg$B. De este modo, demostramos que $(I \land fv \leq cota)$ implica $\neg B$

\newpage

\section{Ap\'endice I: demostraciones formales}

    \subsection{vueloEscalerado}

        \textbf{Pc $\implica$ I}

        $ Pc \implica i == 1 \land \longitud{vueloRealizado(this)} > 2 \land escalerado == (dirx \neq 0 \land diry \neq 0) \\ \land dirx == \prm{vueloRealizado(this)_0} - \prm{vueloRealizado(this)_2} \\ \land diry == \sgd{vueloRealizado(this)_0} - \sgd{vueloRealizado(this)_2} \\
        \explicacion{vueloRealizado(this) - 2 $\geq$ 1} \\
        \implica \bm{1 \leq i \leq \longitud{vueloRealizado(this)} -2} \land escalerado == (dirx \neq 0 \land diry \neq 0) \\ \land dirx == \prm{vueloRealizado(this)_0} - \prm{vueloRealizado(this)_2} \\ \land diry == \sgd{vueloRealizado(this)_0} - \sgd{vueloRealizado(this)_2} \\
        \explicacion{$(escalerado == p) == (escalerado == p \land true)$} \\
        \implica 1 \leq i \leq \longitud{vueloRealizado(this)} -2 \ \land \ escalerado == (dirx \neq 0 \land diry \neq 0 \\ \bm{\land (\prm{vueloRealizado(this)_0} - \prm{vueloRealizado(this)_2}) == dirx \\ \land \ (\sgd{vueloRealizado(this)_0} - \sgd{vueloRealizado(this)_2}) == diry)} \\
        \explicacion{dado $ i == 1$, $[0..i) == [0]$} \\
        \implica 1 \leq i \leq \longitud{vueloRealizado(this)} -2 \ \land \ escalerado == (dirx \neq 0 \land diry \neq 0 \ \land \ \bm{(\forall j \leftarrow [0..i))} \\ (\prm{vueloRealizado(this)_j} - \prm{vueloRealizado(this)_{j+2}} == dirx \\ \land \ \sgd{vueloRealizado(this)_j} - \sgd{vueloRealizado(this)_{j+2}} == diry) \\
        \explicacion{por definicion} \\
        \implica I $

        \bigskip
        \textbf{(I $\land \neg$ B) $\implica$ Qc}

        $I \land \neg B \\
        \implica (\neg escalerado \ \lor \ i \geq \longitud{vueloRealizado(this)} - 2) \land 1 \leq i \leq \longitud{vueloRealizado(this)} -2 \ \land \\ escalerado == (dirx \neq 0 \land(dir) \neq 0 \ \land \ (\forall j \leftarrow [0..i)) (\prm{vueloRealizado(this)_j} - \prm{vueloRealizado(this)_{j+2}} \\ == dirx \ \land \ \sgd{vueloRealizado(this)_j} - \sgd{vueloRealizado(this)_{j+2}} == diry) \\
        \\
        \explicacion{$ (p \lor q) \land r \implica (p \land r) \lor (q \land r) $, y $ a \geq b \land a \leq b \implica a == b $} \\
        \\
        \implica (\bm{\neg escalerado} \land 1 \leq i \leq \longitud{vueloRealizado(this)} -2 \ \land \\ escalerado == (dirx \neq 0 \land diry \neq 0 \land \ (\forall j \leftarrow [0..i)) (\prm{vueloRealizado(this)_j} - \prm{vueloRealizado(this)_{j+2}} \\ == dirx \ \land \ \sgd{vueloRealizado(this)_j} - \sgd{vueloRealizado(this)_{j+2}} == diry)) \\
        \lor % espacio para claridad; este o logico es gigante %
        (\bm{i == \longitud{vueloRealizado(this)} - 2} \land \\ escalerado == (dirx \neq 0 \land diry \neq 0 \ \land \ (\forall j \leftarrow [0..i)) (\prm{vueloRealizado(this)_j} - \prm{vueloRealizado(this)_{j+2}} \\ == dirx \ \land \ \sgd{vueloRealizado(this)_j} - \sgd{vueloRealizado(this)_{j+2}} == diry)) $ 

        \bigskip
        Podemos dividir esto en los dos casos contemplados por $\neg B$:

        \bigskip
        Caso $\neg escalerado$

        \bigskip
        $ \neg escalerado \land 1 \leq i \leq \longitud{vueloRealizado(this)} -2 \ \land \\ escalerado == (dirx \neq 0 \land diry \neq 0 \land \ (\forall j \leftarrow [0..i)) \\ (\prm{vueloRealizado(this)_j} - \prm{vueloRealizado(this)_{j+2}} \ == dirx \\ \land \ \sgd{vueloRealizado(this)_j} - \sgd{vueloRealizado(this)_{j+2}} == diry) \\
        \explicacion{$\neg p \implica \neg(p \land q) $} \\
        \implica escalerado == (dirx \neq 0 \land diry \neq 0 \ \land \ ((\forall j \leftarrow [0..i)) \\ (\prm{vueloRealizado(this)_j} - \prm{vueloRealizado(this)_{j+2}} == dirx \\ \land \ \sgd{vueloRealizado(this)_j} - \sgd{vueloRealizado(this)_{j+2}} == diry)) \\ \bm{\land (((\forall j \leftarrow [i..\longitud{vueloRealizado(this)} - 2)) \\ (\prm{vueloRealizado(this)_j} - \prm{vueloRealizado(this)_{j+2}} == dirx \\ \land \ \sgd{vueloRealizado(this)_j} - \sgd{vueloRealizado(this)_{j+2}} == diry)))} \\
        \explicacion{$\forall j \leftarrow [0..i) \land \forall j \leftarrow [i...j) \implica \forall j \leftarrow [0..j) $} \\
        \implica escalerado == (dirx \neq 0 \land diry \neq 0 \ \land \ \bm{((\forall j \leftarrow [0..\longitud{vueloRealizado(this)} - 2))} \\ (\prm{vueloRealizado(this)_j} - \prm{vueloRealizado(this)_{j+2}} == dirx \\ \land \ \sgd{vueloRealizado(this)_j} - \sgd{vueloRealizado(this)_{j+2}} == diry)) \\
        \explicacion{por definicion}$
        $\implica Q_c $

        \bigskip
        Caso $ i \geq \longitud{vueloRealizado(this)} - 2$

        \bigskip
        $ i \geq \longitud{vueloRealizado(this)} - 2 \land 1 \leq i \leq \longitud{vueloRealizado(this)} -2 \ \land \\ escalerado == (dirx \neq 0 \land diry \neq 0 \land \ (\forall j \leftarrow [0..i)) \\ (\prm{vueloRealizado(this)_j} - \prm{vueloRealizado(this)_{j+2}} \ == dirx \\ \land \ \sgd{vueloRealizado(this)_j} - \sgd{vueloRealizado(this)_{j+2}} == diry) \\
        \explicacion{$a \leq b \land b \leq a \implica a == b$} \\
        \implica \bm{i == \longitud{vueloRealizado(this)} - 2} \land \\ escalerado == (dirx \neq 0 \land diry \neq 0 \land \ (\forall j \leftarrow [0..i)) \\ (\prm{vueloRealizado(this)_j} - \prm{vueloRealizado(this)_{j+2}} \ == dirx \\ \land \ \sgd{vueloRealizado(this)_j} - \sgd{vueloRealizado(this)_{j+2}} == diry) \\
        \explicacion{$i == \longitud{vueloRealizado(this)} - 2$} \\
        \implica escalerado == (dirx \neq 0 \land diry \neq 0 \land \ \bm{(\forall j \leftarrow [0..\longitud{vueloRealizado(this)} - 2))} \\ (\prm{vueloRealizado(this)_j} - \prm{vueloRealizado(this)_{j+2}} \ == dirx \\ \land \ \sgd{vueloRealizado(this)_j} - \sgd{vueloRealizado(this)_{j+2}} == diry) \\
        \explicacion{por definicion} \\
        \implica Q_c $

        \bigskip
        Considerando ambos casos:

        \bigskip
        $ Q_c \lor Q_c \implica Q_c $ \\
        \explicacion{Q.E.D.}

        \bigskip
        \textbf{I $\land$ fv $\leq$ cota $\implica \neg$ B}

        $I \land f_v \leq cota \\
        \implica \longitud{vueloRealizado(this)} - i - 2 \leq 0 \ \land \ 1 \leq i \leq \longitud{vueloRealizado(this)} -2 \\ \land \ escalerado == (dirx \neq 0 \land diry \neq 0 \ \land \ (\forall j \leftarrow [0..i) \\ (\prm{vueloRealizado(this)_j} - \prm{vueloRealizado(this)_{j+2}} == dirx \\ \land \ \sgd{vueloRealizado(this)_j} - \sgd{vueloRealizado(this)_{j+2}} == diry) \\
        \explicacion{$p \land q \implica p$}\\
        \implica \longitud{vueloRealizado(this)} - i - 2 \leq 0 \\
        \explicacion{por inecuaciones}\\
        \implica \longitud{vueloRealizado(this)} - 2 \bm{\leq i} \\
        \explicacion{$p \implica p \lor q$} \\
        \implica \longitud{vueloRealizado(this)} - 2 \leq i \lor \bm{\neg escalerado} \\
        \explicacion{por definicion} \\
        \implica \neg B $

    \subsection{listoParaCosechar}

        \textbf{Pc $\implica$ I}

        $P_c \\
        \implica cantCosechables == 0 \land i == 0 \\
        \explicacion{$x \leftarrow [0], y \leftarrow [], \longitud{[]} == 0$} \\
        \implica i == 0 \land cantCosechables == \bm{|[1| x \leftarrow [0..i) \\ y \leftarrow [0..min(\sgd{dimensiones(campo(this))}, i - x * \sgd{dimensiones(campo(this))})), \\ estadoDelCultivo((x,y), this) == ListoParaCosechar]|} \\
        \explicacion{$i = 0 \implica i = 0 \lor i > 0$} \\
        \implica \bm{0 \leq i \leq (\prm{dimensiones(campo(this))} \times \sgd{dimensiones(campo(this))}} \ \land \ cantCosechables == \\ |[1| x \leftarrow [0..i \div \sgd{dimensiones(campo(this))}], \\ y \leftarrow [0..min(\sgd{dimensiones(campo(this))}, i - x * \sgd{dimensiones(campo(this))})), \\ estadoDelCultivo((x,y), this) == ListoParaCosechar]| \\
        \explicacion{por definicion} \\
        \implica I $

        \bigskip
        \textbf{(I $\land \neg$ B) $\implica$ Qc}

        $(I \land \neg B) \\
        \implica (\prm{dimensiones(campo(this))} \times \sgd{dimensiones(campo(this))}) \leq i \\ \land 0 \leq i \leq (\prm{dimensiones(campo(this))} \times \sgd{dimensiones(campo(this))} \land cantCosechables == \\ | [1| x \leftarrow [0..i \div \sgd{dimensiones(campo(this))}], \\ y \leftarrow [0..min(\sgd{dimensiones(campo(this))}, i - x * \sgd{dimensiones(campo(this))})), \\ estadoDelCultivo((x,y), this) == ListoParaCosechar]| \\
        \explicacion{$ a \geq b \land a \leq b \implica a == b $} \\
        \implica \bm{(\prm{dimensiones(campo(this))} \times \sgd{dimensiones(campo(this))}) == i} \land cantCosechables == \\ | [1| x \leftarrow [0..i \div \sgd{dimensiones(campo(this))}], \\ y \leftarrow [0..min(\sgd{dimensiones(campo(this))}, i - x * \sgd{dimensiones(campo(this))})), \\ estadoDelCultivo((x,y), this) == ListoParaCosechar]| $

        \bigskip
        \explicacion{$i \div \sgd{dimensiones(campo(this))} == \prm{dimensiones(campo(this))}$}

        \bigskip
        $\implica (\prm{dimensiones(campo(this))}\times \sgd{dimensiones(campo(this))}) == i \land cantCosechables == \\ | [1| x \leftarrow [0..\bm{\prm{dimensiones(campo(this))}}], \\ y \leftarrow [0..min(\sgd{dimensiones(campo(this))}, i - x * \sgd{dimensiones(campo(this))})), \\ estadoDelCultivo((x,y), this) == ListoParaCosechar]| $

        \bigskip
        \explicacion{$x == \prm{dimensiones(campo(this))} \implica i - x * \sgd{dimensiones(campo(this))}) == 0 $}

        \explicacion{$(\forall x \leftarrow [0..\prm{dimensiones(campo(this))})) \ \sgd{dimensiones(campo(this))} \leq i - x * \sgd{dimensiones(campo(this))}) $}

        \bigskip
        $\implica (\prm{dimensiones(campo(this))} \times \sgd{dimensiones(campo(this))}) == i \land cantCosechables == \\ | [1| x \leftarrow [0..\prm{dimensiones(campo(this))}), y \leftarrow [0..\bm{\sgd{dimensiones(campo(this))}}), \\ estadoDelCultivo((x,y), this) == ListoParaCosechar]| \\
        \explicacion{por definicion} \\
        \implica Q_c $

        \bigskip
        \textbf{I $\land$ fv $\leq$ cota $\implica \neg$B}

        $(I \land fv < cota) \\
        \implica \prm{dimensiones(campo(this))} \times \sgd{dimensiones(campo(this))} - i \leq 0 \\ \land 0 \leq i \leq (\prm{dimensiones(campo(this))} \times \sgd{dimensiones(campo(this))}) \\ \land | [1| x \leftarrow [0..i \div \sgd{dimensiones(campo(this))}], \\ y \leftarrow [0..min(\sgd{dimensiones(campo(this))}, i - x * \sgd{dimensiones(campo(this))})), \\ estadoDelCultivo((x,y), this) == ListoParaCosechar]| == cantCosechables \\
        \explicacion{$p \land q \implica p$} \\
        \implica \prm{dimensiones(campo(this))} \times \sgd{dimensiones(campo(this))} - i \leq 0 \\
        \explicacion{por inecuaciones} \\
        \implica \prm{dimensiones(campo(this))} \times \sgd{dimensiones(campo(this))} \bm{\leq i} \\
        \explicacion{por definicion} \\
        \implica \neg B $

\end{document} %Termin�!
